% Cadabra notebook version 1.2
\documentclass[11pt]{article}
\usepackage[textwidth=460pt, textheight=660pt]{geometry}
\usepackage[usenames,dvipsnames]{color}
\usepackage{amssymb}
\usepackage[parfill]{parskip}
\usepackage{breqn}
\usepackage{tableaux}
\def\specialcolon{\mathrel{\mathop{:}}\hspace{-.5em}}
\renewcommand{\bar}[1]{\overline{#1}}
\begin{document}
{\color[named]{Blue}\begin{verbatim}
{M,N,K,L,P,Q,R,S,T,U,V, M#}::Indices(name=sl5fund, position=independent).
{A,B,C,D,E,F,G,H, A#}::Indices(name=sl5flat, position=independent).
{\alpha,\beta,\gamma,\delta,\alpha#}::Indices(name=sl5rep10, position=independent).
{\mu,\nu,\mu#,\nu#}::Indices(name=gl7curved, position=independent).
{a,b,c,d,a#,b#}::Indices(name=gl7flat, position=independent).

{M,N,K,L,P,Q,R,S,T,U,V, M#}::Integer(1..5).
{A,B,C,D,E,F,G,H, A#}::Integer(1..5).
{\alpha,\beta,\gamma,\delta,\alpha#}::Integer(1..10).
{\mu,\nu,\mu#,\nu#}::Integer(1..7).
{a,b,c,d,a#,b#}::Integer(1..7).

T^{\alpha}_{M N}::TableauSymmetry(shape={1,1}, indices={1,2}).
T^{\alpha}_{A B}::TableauSymmetry(shape={1,1}, indices={1,2}).
A_{\mu}^{M N}::TableauSymmetry(shape={1,1}, indices={1,2}).
M^{M N K L}::TableauSymmetry(shape={1,1}, indices={0,1}, shape={1,1}, indices={2,3}).
F_{\nu \mu}^{M N}::TableauSymmetry(shape={1,1}, indices={0,1}, shape={1,1}, indices={2,3}).
F_{\nu \mu}^{A B}::TableauSymmetry(shape={1,1}, indices={0,1}, shape={1,1}, indices={2,3}).
F_{M N a}^{b}::TableauSymmetry(shape={1,1}, indices={0,1}).
F_{A B a}^{b}::TableauSymmetry(shape={1,1}, indices={0,1}).
F_{A B C}^{D}::TableauSymmetry(shape={1,1}, indices={0,1}).
G_{A B C}^{D}::TableauSymmetry(shape={1,1}, indices={0,1}).
F_{M N}::AntiSymmetric.
\Lambda^{M N}::AntiSymmetric.
\Lambda^{A B}::AntiSymmetric.
g_{\mu \nu}::Symmetric.

\partial{#}::Derivative.
D{#}::Derivative.
d{#}::PartialDerivative.
\delta{#}::KroneckerDelta.

{\delta_{A}^{B}, E_{A}^{K}, E^{B}_{L}, M^{L M N P}, T^{\alpha}_{K M},  \partial_{\alpha}{\partial_{\beta}{\xi^{\mu}}}, \partial_{\alpha}{E_{A}^{K}}}::SortOrder.

\partial{#}::WeightInherit(label=all, type=Additive):
\nabla{#}::WeightInherit(label=all, type=Additive):
d{#}::WeightInherit(label=all, type=Additive):

{\xi^{\mu}, \xi_{\mu}}::Weight(label=xi, value=1):
\end{verbatim}}
% Begin TeX comment
Assigning list property Indices to $\{M,\; N,\; K,\; L,\; P,\; Q,\; R,\; S,\; T,\; U,\; V,\; M\#\}$.
\\
% End TeX comment
% Begin TeX comment
Assigning list property Indices to $\{A,\; B,\; C,\; D,\; E,\; F,\; G,\; H,\; A\#\}$.
\\
% End TeX comment
% Begin TeX comment
Assigning list property Indices to $\{\alpha,\; \beta,\; \gamma,\; \delta,\; \alpha\#\}$.
\\
% End TeX comment
% Begin TeX comment
Assigning list property Indices to $\{\mu,\; \nu,\; \mu\#,\; \nu\#\}$.
\\
% End TeX comment
% Begin TeX comment
Assigning list property Indices to $\{a,\; b,\; c,\; d,\; a\#,\; b\#\}$.
\\
% End TeX comment
% Begin TeX comment
Assigning property Integer to $M$, $N$, $K$, $L$, $P$, $Q$, $R$, $S$, $T$, $U$, $V$, $M\#$.
\\
% End TeX comment
% Begin TeX comment
Assigning property Integer to $A$, $B$, $C$, $D$, $E$, $F$, $G$, $H$, $A\#$.
\\
% End TeX comment
% Begin TeX comment
Assigning property Integer to $\alpha$, $\beta$, $\gamma$, $\delta$, $\alpha\#$.
\\
% End TeX comment
% Begin TeX comment
Assigning property Integer to $\mu$, $\nu$, $\mu\#$, $\nu\#$.
\\
% End TeX comment
% Begin TeX comment
Assigning property Integer to $a$, $b$, $c$, $d$, $a\#$, $b\#$.
\\
% End TeX comment
% Begin TeX comment
Assigning property TableauSymmetry to ${T}^{\alpha}\,_{M N}$.
\\
% End TeX comment
% Begin TeX comment
Assigning property TableauSymmetry to ${T}^{\alpha}\,_{A B}$.
\\
% End TeX comment
% Begin TeX comment
Assigning property TableauSymmetry to ${A}_{\mu}\,^{M N}$.
\\
% End TeX comment
% Begin TeX comment
Assigning property TableauSymmetry to ${M}^{M N K L}$.
\\
% End TeX comment
% Begin TeX comment
Assigning property TableauSymmetry to ${F}_{\nu \mu}\,^{M N}$.
\\
% End TeX comment
% Begin TeX comment
Assigning property TableauSymmetry to ${F}_{\nu \mu}\,^{A B}$.
\\
% End TeX comment
% Begin TeX comment
Assigning property TableauSymmetry to ${F}_{M N a}\,^{b}$.
\\
% End TeX comment
% Begin TeX comment
Assigning property TableauSymmetry to ${F}_{A B a}\,^{b}$.
\\
% End TeX comment
% Begin TeX comment
Assigning property TableauSymmetry to ${F}_{A B C}\,^{D}$.
\\
% End TeX comment
% Begin TeX comment
Assigning property TableauSymmetry to ${G}_{A B C}\,^{D}$.
\\
% End TeX comment
% Begin TeX comment
Assigning property AntiSymmetric to ${F}_{M N}$.
\\
% End TeX comment
% Begin TeX comment
Assigning property AntiSymmetric to ${\Lambda}^{M N}$.
\\
% End TeX comment
% Begin TeX comment
Assigning property AntiSymmetric to ${\Lambda}^{A B}$.
\\
% End TeX comment
% Begin TeX comment
Assigning property Symmetric to ${g}_{\mu \nu}$.
\\
% End TeX comment
% Begin TeX comment
Assigning property Derivative to $\partial{\#}\, $.
\\
% End TeX comment
% Begin TeX comment
Assigning property Derivative to $D{\#}\, $.
\\
% End TeX comment
% Begin TeX comment
Assigning property PartialDerivative to $d{\#}\, $.
\\
% End TeX comment
% Begin TeX comment
Assigning property KroneckerDelta to $\delta(\#)$.
\\
% End TeX comment
% Begin TeX comment
Assigning list property SortOrder to $\{{\delta}_{A}\,^{B},\; {E}_{A}\,^{K},\; {E}^{B}\,_{L},\; {M}^{L M N P},\; {T}^{\alpha}\,_{K M},\; {\partial}_{\alpha}{{\partial}_{\beta}{{\xi}^{\mu}}\, }\, ,\; {\partial}_{\alpha}{{E}_{A}\,^{K}}\, \}$.
\\
% End TeX comment
% Begin TeX comment
Assigning property WeightInherit to $\partial{\#}\, $.
\\
% End TeX comment
% Begin TeX comment
Assigning property WeightInherit to $\nabla(\#)$.
\\
% End TeX comment
% Begin TeX comment
Assigning property WeightInherit to $d{\#}\, $.
\\
% End TeX comment
% Begin TeX comment
Assigning property Weight to ${\xi}^{\mu}$, ${\xi}_{\mu}$.
\\
% End TeX comment
{\color[named]{Blue}\begin{verbatim}
# Here generalized vielbeins E are in SL(5) x R   !
F := e^{\mu}_{a} T^{\alpha}_{M N} \partial_{\alpha}{e_{\mu}^{b}} - 1/7 * \delta_{a}^{b} e^{\mu}_{c} T^{\alpha}_{M N} \partial_{\alpha}{e_{\mu}^{c}};
\end{verbatim}}
% orig
% e^{\mu}_{a} T^{\alpha}_{M N} \partial_{\alpha}{e_{\mu}^{b}} - 1/7 \delta_{a}^{b} e^{\mu}_{c} T^{\alpha}_{M N} \partial_{\alpha}{e_{\mu}^{c}}
% end_orig
\begin{dmath*}[compact, spread=2pt]
F\specialcolon{}= {e}^{\mu}\,_{a} {T}^{\alpha}\,_{M N} {\partial}_{\alpha}{{e}_{\mu}\,^{b}}\,  - \frac{1}{7}\, {\delta}_{a}\,^{b} {e}^{\mu}\,_{c} {T}^{\alpha}\,_{M N} {\partial}_{\alpha}{{e}_{\mu}\,^{c}}\, ;
\end{dmath*}
{\color[named]{Blue}\begin{verbatim}
@vary(%)(
e^{\mu}_{a} -> 1/2 * \Lambda^{M N} T^{\alpha}_{M N} \partial_{\alpha}{e^{\mu}_{a}} - 1/2 * 1/5 * e^{\mu}_{a} T^{\alpha}_{M N} \partial_{\alpha}{\Lambda^{M N}},
e_{\mu}^{b} -> 1/2 * \Lambda^{M N} T^{\alpha}_{M N} \partial_{\alpha}{e_{\mu}^{b}} + 1/2 * 1/5 * e_{\mu}^{b} T^{\alpha}_{M N} \partial_{\alpha}{\Lambda^{M N}},
):
@substitute!!(%)(
T^{\alpha}_{M N} \partial_{\alpha}{e_{\mu}^{b}} -> e_{\mu}^{a} F_{M N a}^{b} + 1/7 * e_{\mu}^{b} F_{M N},
T^{\alpha}_{M N} \partial_{\alpha}{e^{\mu}_{a}} -> - e^{\mu}_{b} F_{M N a}^{b} - 1/7 * e^{\mu}_{a} F_{M N},
):
@distribute!!(%):
@prodrule!!(%):
@distribute!!(%):
@prodrule!!(%):
@distribute!!(%):
@substitute!!(%)(
T^{\alpha}_{M N} \partial_{\alpha}{e_{\mu}^{b}} -> e_{\mu}^{a} F_{M N a}^{b} + 1/7 * e_{\mu}^{b} F_{M N},
T^{\alpha}_{M N} \partial_{\alpha}{e^{\mu}_{a}} -> - e^{\mu}_{b} F_{M N a}^{b} - 1/7 * e^{\mu}_{a} F_{M N},
):
@distribute!!(%):
@substitute!!(%)(
e^{\mu}_{a} e_{\mu}^{b} -> \delta_{a}^{b},
e^{\mu}_{a} e_{\nu}^{a} -> \delta_{\nu}^{\mu},
):
@distribute!!(%):
@eliminate_kr!!(%):
@substitute!!(%)(
F_{M N a}^{a} -> 0,
):
@prodsort!!(%):
@rename_dummies!(%):
@canonicalise!!(%):
@collect_terms!!(%);
\end{verbatim}}
% Begin TeX comment
@prodrule: not applicable.
% End TeX comment
% Begin TeX comment
@distribute: not applicable.
% End TeX comment
% Begin TeX comment
@distribute: not applicable.
% End TeX comment
% orig
%  - 1/2 F_{M N c}^{b} F_{K L a}^{c} \Lambda^{K L} + 1/2 F_{K L a}^{b} T^{\alpha}_{M N} \partial_{\alpha}{\Lambda^{K L}} + 1/2 F_{M N a}^{c} F_{K L c}^{b} \Lambda^{K L} + 1/2 T^{\alpha}_{M N} \Lambda^{K L} \partial_{\alpha}{F_{K L a}^{b}}
% end_orig
\begin{dmath*}[compact, spread=2pt]
F\specialcolon{}=  - \frac{1}{2}\, {F}_{M N c}\,^{b} {F}_{K L a}\,^{c} {\Lambda}^{K L} + \frac{1}{2}\, {F}_{K L a}\,^{b} {T}^{\alpha}\,_{M N} {\partial}_{\alpha}{{\Lambda}^{K L}}\,  + \frac{1}{2}\, {F}_{M N a}\,^{c} {F}_{K L c}\,^{b} {\Lambda}^{K L} + \frac{1}{2}\, {T}^{\alpha}\,_{M N} {\Lambda}^{K L} {\partial}_{\alpha}{{F}_{K L a}\,^{b}}\, ;
\end{dmath*}
{\color[named]{Blue}\begin{verbatim}
diffeoF := @(F) + LiF_{M N a}^{b} - 1/2 * \Lambda^{K L} T^{\alpha}_{K L} \partial_{\alpha}{F_{M N a}^{b}}
- 1/2 * F_{K L a}^{b} T^{\alpha}_{M N} \partial_{\alpha}{\Lambda^{K L}}:
@prodsort!!(%):
@rename_dummies!(%):
@canonicalise!!(%):
@collect_terms!!(%);
@factor_out!(%){\Lambda^{M N}};
\end{verbatim}}
% orig
%  - 1/2 F_{M N c}^{b} F_{K L a}^{c} \Lambda^{K L} + 1/2 F_{M N a}^{c} F_{K L c}^{b} \Lambda^{K L} + 1/2 T^{\alpha}_{M N} \Lambda^{K L} \partial_{\alpha}{F_{K L a}^{b}} + LiF_{M N a}^{b} - 1/2 T^{\alpha}_{K L} \Lambda^{K L} \partial_{\alpha}{F_{M N a}^{b}}
% end_orig
\begin{dmath*}[compact, spread=2pt]
diffeoF\specialcolon{}=  - \frac{1}{2}\, {F}_{M N c}\,^{b} {F}_{K L a}\,^{c} {\Lambda}^{K L} + \frac{1}{2}\, {F}_{M N a}\,^{c} {F}_{K L c}\,^{b} {\Lambda}^{K L} + \frac{1}{2}\, {T}^{\alpha}\,_{M N} {\Lambda}^{K L} {\partial}_{\alpha}{{F}_{K L a}\,^{b}}\,  + {LiF}_{M N a}\,^{b} - \frac{1}{2}\, {T}^{\alpha}\,_{K L} {\Lambda}^{K L} {\partial}_{\alpha}{{F}_{M N a}\,^{b}}\, ;
\end{dmath*}
% orig
% \Lambda^{K L} ( - 1/2 F_{M N c}^{b} F_{K L a}^{c} + 1/2 F_{M N a}^{c} F_{K L c}^{b} + 1/2 T^{\alpha}_{M N} \partial_{\alpha}{F_{K L a}^{b}} - 1/2 T^{\alpha}_{K L} \partial_{\alpha}{F_{M N a}^{b}}) + LiF_{M N a}^{b}
% end_orig
\begin{dmath*}[compact, spread=2pt]
diffeoF\specialcolon{}= {\Lambda}^{K L} ( - \frac{1}{2}\, {F}_{M N c}\,^{b} {F}_{K L a}\,^{c} + \frac{1}{2}\, {F}_{M N a}\,^{c} {F}_{K L c}\,^{b} + \frac{1}{2}\, {T}^{\alpha}\,_{M N} {\partial}_{\alpha}{{F}_{K L a}\,^{b}}\,  - \frac{1}{2}\, {T}^{\alpha}\,_{K L} {\partial}_{\alpha}{{F}_{M N a}\,^{b}}\, ) + {LiF}_{M N a}\,^{b};
\end{dmath*}
\end{document}
